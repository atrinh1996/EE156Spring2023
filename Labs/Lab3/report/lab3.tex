%%%%%%%%%%%%%%%%%%%%%%%%%%%%%%%%%%%%%%%%%%%%%%%%%%%%%%%%%%%%%%%%%%%%%%%%
%
% Template latex file for a common article class for class notes
% and write ups. Additional Configuration and styling options are 
% commented out. ex. Table of Contents and Title page
% 
% Author: Amy Bui
% 
%%%%%%%%%%%%%%%%%%%%%%%%%%%%%%%%%%%%%%%%%%%%%%%%%%%%%%%%%%%%%%%%%%%%%%%%
 
\documentclass[12pt]{article}
\usepackage[utf8]{inputenc}
\usepackage{parskip}
\usepackage{tabularx}
\usepackage{array}
\usepackage{appendix}
% \usepackage[showframe=true]{geometry}
\usepackage{changepage}
% \usepackage{csvsimple}
% \usepackage[framemethod=tikz]{mdframed}
% \usepackage[color, leftbars]{changebar}
% \usepackage[inkscapeformat=png]{svg}
% \usepackage{svg}
% \usepackage[inkscape={/Applications/Inkscape.app/Contents/Resources/bin/inkscape -z -C}]{svg}

% Important Configurations
 
%%%%%%%%%%%%%%%%%%%%%%%%%%%%%%%%%%%%%%%%%%%%%%%%%%%%%%%%%%%%%%%%%%%%%%%%
% Reduce margin
%
% \addtolength{\oddsidemargin}{-.85in}
% \addtolength{\evensidemargin}{-.85in}
% \addtolength{\textwidth}{1in}

% \addtolength{\topmargin}{-.85in}
% \addtolength{\textheight}{1in}

% Page format commands:
% Override normal article margins,
% making the margins smaller
\setlength{\textwidth}{6.5in}
\setlength{\textheight}{9in}
\setlength{\oddsidemargin}{0in}
\setlength{\evensidemargin}{0in}
\setlength{\topmargin}{-0.6in}

\setlength{\parindent}{0pt}
%%%%%%%%%%%%%%%%%%%%%%%%%%%%%%%%%%%%%%%%%%%%%%%%%%%%%%%%%%%%%%%%%%%%%%%%


%%%%%%%%%%%%%%%%%%%%%%%%%%%%%%%%%%%%%%%%%%%%%%%%%%%%%%%%%%%%%%%%%%%%%%%%
% Math Symbols
\usepackage{mathtools}
\usepackage{amssymb}
% \usepackage{epsfig}
\usepackage{amsmath,amsthm}
\usepackage{amscd,amsxtra,latexsym}


% add floor and ceiling symbol. Usage: \ceil*{}, \floor*{}
\DeclarePairedDelimiter\ceil{\lceil}{\rceil}
\DeclarePairedDelimiter\floor{\lfloor}{\rfloor}

% multiset \langle ... \rangle
\def\multiset#1#2{\ensuremath{\left(\kern-.3em\left(\genfrac{}{}{0pt}{}{#1}{#2}\right)\kern-.3em\right)}}



%%%%%%%%%%%%%%%%%%%%%%%%%%%%%%%%%%%%%%%%%%%%%%%%%%%%%%%%%%%%%%%%%%%%%%%%

%%%%%%%%%%%%%%%%%%%%%%%%%%%%%%%%%%%%%%%%%%%%%%%%%%%%%%%%%%%%%%%%%%%%%%%%
% Code Sample Styling

% use \lstinline! xxx ! or \begin{lstlisting} ... \end{lstlisting}
\usepackage{listings}

\usepackage{color}
\definecolor{light-gray}{gray}{0.97} % shade of grey
\definecolor{dkgreen}{rgb}{0,0.6,0}
\definecolor{gray}{rgb}{0.5,0.5,0.5}
\definecolor{mauve}{rgb}{0.58,0,0.82}

% \begin{lstlisting}[...] ... \end{lstlisting}
\lstset{frame=none,
    language=Verilog,
    aboveskip=3mm,
    belowskip=3mm,
    stepnumber=0, % set to 0 if you don't like line nums
    showstringspaces=false,
    columns=flexible,
    basicstyle={\small\ttfamily},
    numbers=left,
    numberstyle=\color{black},
    keywordstyle=\color{blue},
    commentstyle=\color{dkgreen},
    stringstyle=\color{mauve},
    backgroundcolor=\color{light-gray},
    breaklines=true,
    breakatwhitespace=false,
    tabsize=2
}

% \newcommand\mylstcaption{}

% \mdfdefinestyle{mymdstyle}{
% hidealllines=true,
% middleextra={
%   \node[anchor=west] at (O|-P)
%     {\lstlistingname~\thelstlisting\  (Cont.):~\mylstcaption};},
% secondextra={
%   \node[anchor=west] at (O|-P)
%     {\lstlistingname~\thelstlisting\  (Cont.):~\mylstcaption};},
% splittopskip=2\baselineskip
% }

% \surroundwithmdframed[style=mymdstyle]{lstlisting}
% \newmdenv[style=mymdstyle]{mdlisting}



%%%%%%%%%%%%%%%%%%%%%%%%%%%%%%%%%%%%%%%%%%%%%%%%%%%%%%%%%%%%%%%%%%%%%%%%

%%%%%%%%%%%%%%%%%%%%%%%%%%%%%%%%%%%%%%%%%%%%%%%%%%%%%%%%%%%%%%%%%%%%%%%%
\usepackage{xcolor}
%% https://tex.stackexchange.com/questions/401750/quick-and-short-command-for-coloring-one-word
\newcommand\shorthandon{\catcode`@=\active \catcode`^=\active \catcode`*=\active }
\newcommand\shorthandoff{\catcode`@=12 \catcode`^=7 \catcode`*=12 }
\shorthandon
\def@#1@{\textcolor{red}{#1}}%
\def^#1^{\textcolor{blue}{#1}}%
\def*#1{\string#1}
\shorthandoff
%% useage: \textcolor{red}{text here}
% \shorthandon
% This is a @test@ of the ^emergency^ bro*@dcast system.
% \shorthandoff
%%%%%%%%%%%%%%%%%%%%%%%%%%%%%%%%%%%%%%%%%%%%%%%%%%%%%%%%%%%%%%%%%%%%%%%%


%%%%%%%%%%%%%%%%%%%%%%%%%%%%%%%%%%%%%%%%%%%%%%%%%%%%%%%%%%%%%%%%%%%%%%%%

%Commands below change page margins (this much space at the titlepage, etc)
\newlength{\toppush}
\setlength{\toppush}{2\headheight}
\addtolength{\toppush}{\headsep}

% Section header Styling
% The commands below change the bold text where it says "Section" into "Question"
% \usepackage{titlesec}
% \titleformat{\section}
% {\normalfont\Large\bfseries}{Question~\thesection:}{1em}{}

% I added this command below to chance "subsections numbers" to be "Question [subsection number]" -AB 1/31/2021
% \titleformat{\subsection}
% {\normalfont\bfseries}{\thesubsection:}{1em}{}

% Page head Styling
% Name and subject of the class
\def\subjnum{EE 156}          % Class Number
\def\subjname{Advance Topics in Computer Architecture}       % Class Name

% Name of the student, university name and which semester
\def\doheading#1#2#3{\vfill\eject\vspace*{-\toppush}%
  \vbox{\hbox to\textwidth{{\bf} \subjnum: \subjname \hfil Amy Bui}%
    \hbox to\textwidth{{\bf} Tufts University, Spring 2023 \hfil#3\strut}%
    \hrule}}

%Command for the title of the document (Homework 0)
\newcommand{\htitle}[1]{\vspace*{1.25ex plus 1ex minus 0ex}%
\begin{center}
    {\large\bf #1}
\end{center}} 
%%%%%%%%%%%%%%%%%%%%%%%%%%%%%%%%%%%%%%%%%%%%%%%%%%%%%%%%%%%%%%%%%%%%%%%%



%%%%%%%%%%%%%%%%%%%%%%%%%%%%%%%%%%%%%%%%%%%%%%%%%%%%%%%%%%%%%%%%%%%%%%%%
% Misc
\usepackage{graphicx} % graphics
\usepackage{enumitem} % listing style (bullet lists)

% below helps with trying to get figures in a row
\usepackage{caption}
\usepackage{subcaption}

% hyperlink styling
% use \href{} and \url{}, and colors table of contents links
% use \href{} and \url{}
% \label{sec:name}
% \hyperref[label]{text}
\usepackage{hyperref}
\hypersetup{
    colorlinks=true,
    linkcolor=blue, % was previously black
    filecolor=magenta,
    urlcolor=blue,
    pdftitle={Template}
}
\urlstyle{same}

% A command for primes (')
\newcommand{\p}%
    {\ensuremath{^{\prime}}}

% a command for double primes ('')
\newcommand{\pp}%
    {\ensuremath{^{\prime \prime}}}

% A command for the Kleene star
\newcommand{\str}%
    {\ensuremath{^{\star}}}

% a command for the double star
\newcommand{\sstr}%
    {\ensuremath{^{\star\star}}}
%%%%%%%%%%%%%%%%%%%%%%%%%%%%%%%%%%%%%%%%%%%%%%%%%%%%%%%%%%%%%%%%%%%%%%%%

% Options for title page, use \maketitle in document
% \author{Amy Bui}
% \title{COMP160 - Algorithms: Class Notes and Practice}


\begin{document}
%% create title page
% \title{(g)ROOT \\ Language Reference Manual}
% \author{}
% \date{\today}
% \maketitle

\doheading{2}{title}{Lab 3 (*used 3 late tokens)}

    %%%%%%%%%%%%%%%%%%%%%%%%%%%%%%%%%%%%%%%%%%%%%%%%%%%%%%%%%%%%%%%%%%%%%%%%
    % Table of Contents
    \setcounter{tocdepth}{2}
    \tableofcontents
    % \pagebreak
    %%%%%%%%%%%%%%%%%%%%%%%%%%%%%%%%%%%%%%%%%%%%%%%%%%%%%%%%%%%%%%%%%%%%%%%%

    \begin{thebibliography}{1}
        \bibitem[1]{sniper}\href{https://snipersim.org/w/The_Sniper_Multi-Core_Simulator}{The Sniper Multi-Core Simulator}

        \bibitem[2]{parallel}O. Tange (2011): \href{https://www.gnu.org/software/parallel/parallel_tutorial.html}{GNU Parallel}  - The Command-Line Power Tool

        \bibitem[3]{splash2}S. C. Woo, M. Ohara, E. Torrie, J. P. Singh and A. Gupta, \href{https://ieeexplore-ieee-org.ezproxy.library.tufts.edu/stamp/stamp.jsp?tp=&arnumber=524546}{The SPLASH-2 Programs: Characterization and Methodological Considerations}, Proceedings 22nd Annual International Symposium on Computer Architecture, Santa Margherita Ligure, Italy, 1995, pp. 24-36

        \bibitem[4]{book}John L. Hennessy and David A. Patterson. 2017. Computer Architecture, Sixth Edition: A Quantitative Approach. Morgan Kaufmann Publishers Inc., San Francisco, CA, USA.

        \bibitem[5]{mcfarling}McFarling, Scott, 1993. \href{https://www.hpl.hp.com/techreports/Compaq-DEC/WRL-TN-36.pdf}{Combining Branch Predictors}, Technical Note TN-36, Western Research Laboratory. Digital Equipment Corp., Palo Alto, CA.
    \end{thebibliography}

    % SWEEP
    
    \begin{table}[hbt!] 
        \begin{center} 
            \begin{tabular}{c||c}
                \begin{tabular}{|l|}
                    \hline
                    \textbf{Benchmark} \\ 
                    \hline 
                    \hline
                    \texttt{cholesky} \\ 
                    \texttt{fmm}\\
                    \texttt{lu.cont}\\
                    \texttt{radiosity}\\
                    \texttt{raytrace}\\
                    \hline 
                \end{tabular}
                & 
                \begin{tabular}{|l|}
                    \hline
                    \textbf{PHT index bits (gshare)} \\ 
                    \hline 
                    \hline
                    0 (default branch predictor) \\
                    4 \\ 
                    8 \\ 
                    16 \\
                    \hline 
                \end{tabular}
            \end{tabular}
            \caption{Configuration parameters and values for PHT index bits for the gshare branch predictor swept in the experiment. The default branch predictor is pentium.}
            \label{table:configurations}
        \end{center}
    \end{table}

    % TOPOLOGY
    \begin{figure}[hbt!] 
        \centering
        \includegraphics[width=0.7\textwidth]{./images/topo.png}
        \caption{Topology for all five benchmark tests where only the number of PHT index bits were varied (all cache sizes remained consistent through every simulation). All benchmarks were run in \texttt{Sniper-7.3} with the \texttt{gainestown} configuration using the \texttt{--viz} and \texttt{--roi} options.}
        \label{topology} 
    \end{figure}


    \clearpage

    \section{Intro}
    \label{intro}
        % Tomasulo's algorithm 

        pattern history table (PHT)

    % \clearpage
    %%%%%%%%%%%%%%%%%%%%%%%%%%%%%%%%%%%%%%%%%%%%%%%%%%%%%%%%%%%%%%%%%%%%%%%%
    

    \section{Experimental Setup}
    \label{sec:setup}
        Simulations ran for an x86 architecture simulator, Sniper 7.3 \cite{sniper}. Since this experiment looked to sweep the \texttt{gshare} branch predictor performance through PHT sizes, each simulation was configured the same topology (Figure \ref{topology}). The same default configurations were set in \texttt{gainestown.cfg}. 

        Three PHT sizes were swept for simulations using the \texttt{gshare} branch predictor, $2^4, 2^8, 2^{16}$. The sweep for ``size'' $2^0$ represents simulations that ran the default branch predictor, \texttt{pentium}. \texttt{gshare} was implemented separately according to technical specifications \cite{mcfarling} in C++, and integrated into \texttt{Sniper-7.3} with \texttt{gcc-7.4.0}. There was a total of 26 simulations (see Table \ref{table:configurations}). The different branch predictor configurations were simulated with five \texttt{splash2} benchmarks not previously used: \texttt{cholesky}, \texttt{lu.cont}, \texttt{radiosity}, \texttt{raytrace}, and \texttt{fmm} \cite{splash2}. 
        % \clearpage


        The workloads are briefly described as follows:

        \begin{description}
            \item[\textbf{cholesky}]: The \texttt{cholesky} factors a sparse matrix into the product of a lower triangular matrix and its transpose without globally synchronizing between steps. 
            
            
            
            \item[\textbf{fmm}]: The \texttt{fmm} application implements the adaptive Fast Multipole Method to simulate interactions of systems of N-bodies (particles, galaxies, etc.) in 2D with unstructured communication patterns.
            
            suite of test studies large-scale ocean movements based on currents, and uses 4D array grids and a red-black Gauss-Seidel multigrid equation solver. 
            
            \item[\textbf{lu.cont}]: The \texttt{lu.cont} factors a dense matrix into the product of a lower and upper triangular matrix, exploiting temporal locality on submatrix elements. Blocks are allocated sequentially and locally to processors that own them in order to improve the spatial locality. 
            
            suite uses an iterative radix sort algorithm that generates histograms and has each processor permute array index keys, a process that depends on processors communicating in order to determine keys through writes.
            
            \item[\textbf{radiosity}]: The \texttt{radiosity} application computes the equilibrium distribution of light in a scene using the iterative hierarchical diffuse radiosity method using the light transport interactions and subdivisions in polygons. This application has highly irregular computation structure and data structure accesses.  
            
            \item[\textbf{raytrace}]: The \texttt{raytrace} renders a 3D scene using ray tracing through each pixel in the image plane, reflecting them off objects in unpredictable and multiple ways; therefore, data structure access patterns are also unpredictable.
        \end{description}





        All the simulations ran concurrently using bash script(s) and GNU \texttt{parallel} shell tool \cite{parallel}, and post processing of the data were handled with python (v2.7) and bash scripts (included separately). Simulations ran on a python virtual environment and in a detached \texttt{tmux} session, due to long duration of the experiments. Sniper provided data processing tools used were: \texttt{gen\_topology.py}, \texttt{cpi-stack.py}, and \texttt{mcpat.py}. 
        
        

    \clearpage
    %%%%%%%%%%%%%%%%%%%%%%%%%%%%%%%%%%%%%%%%%%%%%%%%%%%%%%%%%%%%%%%%%%%%%%%%

    \section{Results \& Analysis}


        \subsection{Performance Analysis} %%%%%%%%%%%%%
        \label{sec:performance}

            \begin{figure}[hbt!] 
                \centering 
                \includegraphics[width=0.8\textwidth]{images/PHTvsIPC.png}
                \caption{}
                \label{fig:IPC} 
            \end{figure}

            % \textbf{No ILP benefits from OOO:} For workloads likely to not have many dependencies and implemented with little to no parallelism, they see little to no benefits in increasing the number of RS'. 
            % \clearpage
        %%%%%%%%%%%%%%%%%%%%%%%%%%%%%%%%%%%%%%%%%%%%%%%%%%%%



        \subsection{Energy Consumption} %%%%%%%%%%%%%
            \begin{figure}[hbt!] 
                \centering
                \includegraphics[width=0.8\textwidth]{images/PHTvsPower.png}
                \caption{}
                \label{fig:peak-dynamic-power} 
            \end{figure} 

            \begin{figure}[hbt!] 
                \centering
                \includegraphics[width=0.8\textwidth]{images/PHTvsnEDP.png}
                \caption{}
                \label{fig:normalized-EDP} 
            \end{figure}

            % Much like how the IPC is seen to level-off  
    % \clearpage
    %%%%%%%%%%%%%%%%%%%%%%%%%%%%%%%%%%%%%%%%%%%%%%%%%%%%%%%%%%%%%%%%%%%%%%%%


    \section{Conclusion}
    \label{conclusion}

    % Sweeping RS entries show consistent results with what is known in the literature. For most of the workloads, the benefits of ILP

    \clearpage
    %%%%%%%%%%%%%%%%%%%%%%%%%%%%%%%%%%%%%%%%%%%%%%%%%%%%%%%%%%%%%%%%%%%%%%%%

    \section{Appendix: Raw Post Processed Data}
    \label{appendix:raw}

    \subsection{cholesky} %%%%%%%%%%%%%%%%%%%%%%%%%%%%%%%%%%%
        \subsubsection{Power Results} %%%%%%%%%%%%%%%%%%%%%%%%%%%%%%%%%%%

        %%%%%%%%%% Power GRAPH %%%%%%%%%%
        \begin{figure}[hbt!]
            \centering
            \noindent\begin{subfigure}{1\textwidth}
            \includegraphics[width=1\textwidth]{images/Power/cholesky.PNG}
            % \caption{}
            \end{subfigure}%

            \caption{Processor power for various PHT sizes.}
            \label{appfig:power:cholesky}
        \end{figure}
        % \clearpage
        %%%%%%%%%%%%%%%%%%%%%%%%%%%%%%%%%%%

        %%%%%%%%%% Power VALUES %%%%%%%%%%
        \begin{figure}[hbt!]
            \centering
            \noindent\begin{subfigure}{0.75\textwidth}
            \lstinputlisting{../output/cholesky/0/power.out}
            \caption{default}
            \end{subfigure}%

            \noindent\begin{subfigure}{0.75\textwidth}
            \lstinputlisting{../output/cholesky/4/power.out}
            \caption{$2^{4}$ PHT size}
            \end{subfigure}%
        \end{figure}
        \clearpage

        \begin{figure}[hbt!]\ContinuedFloat
            \centering
            \noindent\begin{subfigure}{0.75\textwidth}
            \lstinputlisting{../output/cholesky/8/power.out}
            \caption{$2^{8}$ PHT size}
            \end{subfigure}%

            \noindent\begin{subfigure}{0.75\textwidth}
            \lstinputlisting{../output/cholesky/16/power.out}
            \caption{$2^{16}$ PHT size}
            \end{subfigure}%

            \caption{Specific values for each components' power consumption.}
            \label{appfig:power:cholesky:values}
        \end{figure}
        \clearpage
        %%%%%%%%%%%%%%%%%%%%%%%%%%%%%%%%%%%%%%%%%%%%%%%%%%%%%%%%%%%%%%%%%%%%%%

        \subsubsection{CPI Stacks} %%%%%%%%%%%%%%%%%%%%%%%%%%%%%%%%%%%

        %%%%%%%%%% CPI GRAPH %%%%%%%%%%
        \begin{figure}[hbt!]
            \centering
            \noindent\begin{subfigure}{0.8\textwidth}
            \includegraphics[width=1\textwidth]{../output/cholesky/0/cpi-stack.png}
            \caption{default}
            \end{subfigure}%

            \noindent\begin{subfigure}{0.8\textwidth}
            \includegraphics[width=1\textwidth]{../output/cholesky/0/cpi-stack.png}
            \caption{$2^{4}$ PHT size}
            \end{subfigure}%
        \end{figure}
        \clearpage

        \begin{figure}[hbt!]\ContinuedFloat
            \centering
            \noindent\begin{subfigure}{0.8\textwidth}
            \includegraphics[width=1\textwidth]{../output/cholesky/8/cpi-stack.png}
            \caption{$2^{8}$ PHT size}
            \end{subfigure}%

            \noindent\begin{subfigure}{0.8\textwidth}
            \includegraphics[width=1\textwidth]{../output/cholesky/16/cpi-stack.png}
            \caption{$2^{16}$ PHT size}
            \end{subfigure}%

            \caption{CPI stacks for various PHT sizes.}
            \label{appfig:cpi:cholesky}
        \end{figure}
        \clearpage
        %%%%%%%%%%%%%%%%%%%%%%%%%%%%%%%%%%%

        %%%%%%%%%% CPI VALUES %%%%%%%%%%
        \begin{figure}[hbt!]
            \centering
            \noindent\begin{subfigure}{0.75\textwidth}
            \lstinputlisting{../output/cholesky/0/cpi-stack.out}
            \caption{default}
            \end{subfigure}%

            \noindent\begin{subfigure}{0.75\textwidth}
            \lstinputlisting{../output/cholesky/4/cpi-stack.out}
            \caption{$2^{4}$ PHT size}
            \end{subfigure}%
        \end{figure}
        \clearpage

        \begin{figure}[hbt!]\ContinuedFloat
            \centering
            \noindent\begin{subfigure}{0.75\textwidth}
            \lstinputlisting{../output/cholesky/8/cpi-stack.out}
            \caption{$2^{8}$ PHT size}
            \end{subfigure}%

            \noindent\begin{subfigure}{0.75\textwidth}
            \lstinputlisting{../output/cholesky/16/cpi-stack.out}
            \caption{$2^{16}$ PHT size}
            \end{subfigure}%

            \caption{Specific values for each components' CPI stack.}
            \label{appfig:cpi:cholesky:values}
        \end{figure}
        \clearpage

        %%%%%%%%%%%%%%%%%%%%%%%%%%%%%%%%%%%%%%%%%%%%%%%%%%%%%%%%%%%%%%%%%%%%%%

    \subsection{fmm} %%%%%%%%%%%%%%%%%%%%%%%%%%%%%%%%%%%
        \subsubsection{Power Results} %%%%%%%%%%%%%%%%%%%%%%%%%%%%%%%%%%%

        %%%%%%%%%% Power GRAPH %%%%%%%%%%
        \begin{figure}[hbt!]
            \centering
            \noindent\begin{subfigure}{1\textwidth}
            \includegraphics[width=1\textwidth]{images/Power/fmm.PNG}
            % \caption{}
            \end{subfigure}%

            \caption{Processor power for various PHT sizes.}
            \label{appfig:power:fmm}
        \end{figure}
        % \clearpage
        %%%%%%%%%%%%%%%%%%%%%%%%%%%%%%%%%%%

        %%%%%%%%%% Power VALUES %%%%%%%%%%
        \begin{figure}[hbt!]
            \centering
            \noindent\begin{subfigure}{0.75\textwidth}
            \lstinputlisting{../output/fmm/0/power.out}
            \caption{default}
            \end{subfigure}%

            \noindent\begin{subfigure}{0.75\textwidth}
            \lstinputlisting{../output/fmm/4/power.out}
            \caption{$2^{4}$ PHT size}
            \end{subfigure}%
        \end{figure}
        \clearpage

        \begin{figure}[hbt!]\ContinuedFloat
            \centering
            \noindent\begin{subfigure}{0.75\textwidth}
            \lstinputlisting{../output/fmm/8/power.out}
            \caption{$2^{8}$ PHT size}
            \end{subfigure}%

            \noindent\begin{subfigure}{0.75\textwidth}
            \lstinputlisting{../output/fmm/16/power.out}
            \caption{$2^{16}$ PHT size}
            \end{subfigure}%

            \caption{Specific values for each components' power consumption.}
            \label{appfig:power:fmm:values}
        \end{figure}
        \clearpage
        %%%%%%%%%%%%%%%%%%%%%%%%%%%%%%%%%%%%%%%%%%%%%%%%%%%%%%%%%%%%%%%%%%%%%%

        \subsubsection{CPI Stacks} %%%%%%%%%%%%%%%%%%%%%%%%%%%%%%%%%%%

        %%%%%%%%%% CPI GRAPH %%%%%%%%%%
        \begin{figure}[hbt!]
            \centering
            \noindent\begin{subfigure}{0.8\textwidth}
            \includegraphics[width=1\textwidth]{../output/fmm/0/cpi-stack.png}
            \caption{default}
            \end{subfigure}%

            \noindent\begin{subfigure}{0.8\textwidth}
            \includegraphics[width=1\textwidth]{../output/fmm/0/cpi-stack.png}
            \caption{$2^{4}$ PHT size}
            \end{subfigure}%
        \end{figure}
        \clearpage

        \begin{figure}[hbt!]\ContinuedFloat
            \centering
            \noindent\begin{subfigure}{0.8\textwidth}
            \includegraphics[width=1\textwidth]{../output/fmm/8/cpi-stack.png}
            \caption{$2^{8}$ PHT size}
            \end{subfigure}%

            \noindent\begin{subfigure}{0.8\textwidth}
            \includegraphics[width=1\textwidth]{../output/fmm/16/cpi-stack.png}
            \caption{$2^{16}$ PHT size}
            \end{subfigure}%

            \caption{CPI stacks for various PHT sizes.}
            \label{appfig:cpi:fmm}
        \end{figure}
        \clearpage
        %%%%%%%%%%%%%%%%%%%%%%%%%%%%%%%%%%%

        %%%%%%%%%% CPI VALUES %%%%%%%%%%
        \begin{figure}[hbt!]
            \centering
            \noindent\begin{subfigure}{0.75\textwidth}
            \lstinputlisting{../output/fmm/0/cpi-stack.out}
            \caption{default}
            \end{subfigure}%

            \noindent\begin{subfigure}{0.75\textwidth}
            \lstinputlisting{../output/fmm/4/cpi-stack.out}
            \caption{$2^{4}$ PHT size}
            \end{subfigure}%
        \end{figure}
        \clearpage

        \begin{figure}[hbt!]\ContinuedFloat
            \centering
            \noindent\begin{subfigure}{0.75\textwidth}
            \lstinputlisting{../output/fmm/8/cpi-stack.out}
            \caption{$2^{8}$ PHT size}
            \end{subfigure}%

            \noindent\begin{subfigure}{0.75\textwidth}
            \lstinputlisting{../output/fmm/16/cpi-stack.out}
            \caption{$2^{16}$ PHT size}
            \end{subfigure}%

            \caption{Specific values for each components' CPI stack.}
            \label{appfig:cpi:fmm:values}
        \end{figure}
        \clearpage
        %%%%%%%%%%%%%%%%%%%%%%%%%%%%%%%%%%%%%%%%%%%%%%%%%%%%%%%%%%%%%%%%%%%%%%

    \subsection{lu.cont} %%%%%%%%%%%%%%%%%%%%%%%%%%%%%%%%%%%
        \subsubsection{Power Results} %%%%%%%%%%%%%%%%%%%%%%%%%%%%%%%%%%%

        %%%%%%%%%% Power GRAPH %%%%%%%%%%
        \begin{figure}[hbt!]
            \centering
            \noindent\begin{subfigure}{1\textwidth}
            \includegraphics[width=1\textwidth]{images/Power/lu.cont.PNG}
            % \caption{}
            \end{subfigure}%

            \caption{Processor power for various PHT sizes.}
            \label{appfig:power:lu.cont}
        \end{figure}
        % \clearpage
        %%%%%%%%%%%%%%%%%%%%%%%%%%%%%%%%%%%

        %%%%%%%%%% Power VALUES %%%%%%%%%%
        \begin{figure}[hbt!]
            \centering
            \noindent\begin{subfigure}{0.75\textwidth}
            \lstinputlisting{../output/lu.cont/0/power.out}
            \caption{default}
            \end{subfigure}%

            \noindent\begin{subfigure}{0.75\textwidth}
            \lstinputlisting{../output/lu.cont/4/power.out}
            \caption{$2^{4}$ PHT size}
            \end{subfigure}%
        \end{figure}
        \clearpage

        \begin{figure}[hbt!]\ContinuedFloat
            \centering
            \noindent\begin{subfigure}{0.75\textwidth}
            \lstinputlisting{../output/lu.cont/8/power.out}
            \caption{$2^{8}$ PHT size}
            \end{subfigure}%

            \noindent\begin{subfigure}{0.75\textwidth}
            \lstinputlisting{../output/lu.cont/16/power.out}
            \caption{$2^{16}$ PHT size}
            \end{subfigure}%

            \caption{Specific values for each components' power consumption.}
            \label{appfig:power:lu.cont:values}
        \end{figure}
        \clearpage
        %%%%%%%%%%%%%%%%%%%%%%%%%%%%%%%%%%%%%%%%%%%%%%%%%%%%%%%%%%%%%%%%%%%%%%

        \subsubsection{CPI Stacks} %%%%%%%%%%%%%%%%%%%%%%%%%%%%%%%%%%%

        %%%%%%%%%% CPI GRAPH %%%%%%%%%%
        \begin{figure}[hbt!]
            \centering
            \noindent\begin{subfigure}{0.8\textwidth}
            \includegraphics[width=1\textwidth]{../output/lu.cont/0/cpi-stack.png}
            \caption{default}
            \end{subfigure}%

            \noindent\begin{subfigure}{0.8\textwidth}
            \includegraphics[width=1\textwidth]{../output/lu.cont/0/cpi-stack.png}
            \caption{$2^{4}$ PHT size}
            \end{subfigure}%
        \end{figure}
        \clearpage

        \begin{figure}[hbt!]\ContinuedFloat
            \centering
            \noindent\begin{subfigure}{0.8\textwidth}
            \includegraphics[width=1\textwidth]{../output/lu.cont/8/cpi-stack.png}
            \caption{$2^{8}$ PHT size}
            \end{subfigure}%

            \noindent\begin{subfigure}{0.8\textwidth}
            \includegraphics[width=1\textwidth]{../output/lu.cont/16/cpi-stack.png}
            \caption{$2^{16}$ PHT size}
            \end{subfigure}%

            \caption{CPI stacks for various PHT sizes.}
            \label{appfig:cpi:lu.cont}
        \end{figure}
        \clearpage
        %%%%%%%%%%%%%%%%%%%%%%%%%%%%%%%%%%%

        %%%%%%%%%% CPI VALUES %%%%%%%%%%
        \begin{figure}[hbt!]
            \centering
            \noindent\begin{subfigure}{0.75\textwidth}
            \lstinputlisting{../output/lu.cont/0/cpi-stack.out}
            \caption{default}
            \end{subfigure}%

            \noindent\begin{subfigure}{0.75\textwidth}
            \lstinputlisting{../output/lu.cont/4/cpi-stack.out}
            \caption{$2^{4}$ PHT size}
            \end{subfigure}%
        \end{figure}
        \clearpage

        \begin{figure}[hbt!]\ContinuedFloat
            \centering
            \noindent\begin{subfigure}{0.75\textwidth}
            \lstinputlisting{../output/lu.cont/8/cpi-stack.out}
            \caption{$2^{8}$ PHT size}
            \end{subfigure}%

            \noindent\begin{subfigure}{0.75\textwidth}
            \lstinputlisting{../output/lu.cont/16/cpi-stack.out}
            \caption{$2^{16}$ PHT size}
            \end{subfigure}%

            \caption{Specific values for each components' CPI stack.}
            \label{appfig:cpi:lu.cont:values}
        \end{figure}
        \clearpage
        %%%%%%%%%%%%%%%%%%%%%%%%%%%%%%%%%%%%%%%%%%%%%%%%%%%%%%%%%%%%%%%%%%%%%%

    \subsection{radiosity} %%%%%%%%%%%%%%%%%%%%%%%%%%%%%%%%%%%
        \subsubsection{Power Results} %%%%%%%%%%%%%%%%%%%%%%%%%%%%%%%%%%%

        %%%%%%%%%% Power GRAPH %%%%%%%%%%
        \begin{figure}[hbt!]
            \centering
            \noindent\begin{subfigure}{1\textwidth}
            \includegraphics[width=1\textwidth]{images/Power/radiosity.PNG}
            % \caption{}
            \end{subfigure}%

            \caption{Processor power for various PHT sizes.}
            \label{appfig:power:radiosity}
        \end{figure}
        % \clearpage
        %%%%%%%%%%%%%%%%%%%%%%%%%%%%%%%%%%%

        %%%%%%%%%% Power VALUES %%%%%%%%%%
        \begin{figure}[hbt!]
            \centering
            \noindent\begin{subfigure}{0.75\textwidth}
            \lstinputlisting{../output/radiosity/0/power.out}
            \caption{default}
            \end{subfigure}%

            \noindent\begin{subfigure}{0.75\textwidth}
            \lstinputlisting{../output/radiosity/4/power.out}
            \caption{$2^{4}$ PHT size}
            \end{subfigure}%
        \end{figure}
        \clearpage

        \begin{figure}[hbt!]\ContinuedFloat
            \centering
            \noindent\begin{subfigure}{0.75\textwidth}
            \lstinputlisting{../output/radiosity/8/power.out}
            \caption{$2^{8}$ PHT size}
            \end{subfigure}%

            \noindent\begin{subfigure}{0.75\textwidth}
            \lstinputlisting{../output/radiosity/16/power.out}
            \caption{$2^{16}$ PHT size}
            \end{subfigure}%

            \caption{Specific values for each components' power consumption.}
            \label{appfig:power:radiosity:values}
        \end{figure}
        \clearpage
        %%%%%%%%%%%%%%%%%%%%%%%%%%%%%%%%%%%%%%%%%%%%%%%%%%%%%%%%%%%%%%%%%%%%%%

        \subsubsection{CPI Stacks} %%%%%%%%%%%%%%%%%%%%%%%%%%%%%%%%%%%

        %%%%%%%%%% CPI GRAPH %%%%%%%%%%
        \begin{figure}[hbt!]
            \centering
            \noindent\begin{subfigure}{0.8\textwidth}
            \includegraphics[width=1\textwidth]{../output/radiosity/0/cpi-stack.png}
            \caption{default}
            \end{subfigure}%

            \noindent\begin{subfigure}{0.8\textwidth}
            \includegraphics[width=1\textwidth]{../output/radiosity/0/cpi-stack.png}
            \caption{$2^{4}$ PHT size}
            \end{subfigure}%
        \end{figure}
        \clearpage

        \begin{figure}[hbt!]\ContinuedFloat
            \centering
            \noindent\begin{subfigure}{0.8\textwidth}
            \includegraphics[width=1\textwidth]{../output/radiosity/8/cpi-stack.png}
            \caption{$2^{8}$ PHT size}
            \end{subfigure}%

            \noindent\begin{subfigure}{0.8\textwidth}
            \includegraphics[width=1\textwidth]{../output/radiosity/16/cpi-stack.png}
            \caption{$2^{16}$ PHT size}
            \end{subfigure}%

            \caption{CPI stacks for various PHT sizes.}
            \label{appfig:cpi:radiosity}
        \end{figure}
        \clearpage
        %%%%%%%%%%%%%%%%%%%%%%%%%%%%%%%%%%%

        %%%%%%%%%% CPI VALUES %%%%%%%%%%
        \begin{figure}[hbt!]
            \centering
            \noindent\begin{subfigure}{0.75\textwidth}
            \lstinputlisting{../output/radiosity/0/cpi-stack.out}
            \caption{default}
            \end{subfigure}%

            \noindent\begin{subfigure}{0.75\textwidth}
            \lstinputlisting{../output/radiosity/4/cpi-stack.out}
            \caption{$2^{4}$ PHT size}
            \end{subfigure}%
        \end{figure}
        \clearpage

        \begin{figure}[hbt!]\ContinuedFloat
            \centering
            \noindent\begin{subfigure}{0.75\textwidth}
            \lstinputlisting{../output/radiosity/8/cpi-stack.out}
            \caption{$2^{8}$ PHT size}
            \end{subfigure}%

            \noindent\begin{subfigure}{0.75\textwidth}
            \lstinputlisting{../output/radiosity/16/cpi-stack.out}
            \caption{$2^{16}$ PHT size}
            \end{subfigure}%

            \caption{Specific values for each components' CPI stack.}
            \label{appfig:cpi:radiosity:values}
        \end{figure}
        \clearpage
        %%%%%%%%%%%%%%%%%%%%%%%%%%%%%%%%%%%%%%%%%%%%%%%%%%%%%%%%%%%%%%%%%%%%%%

    \subsection{raytrace} %%%%%%%%%%%%%%%%%%%%%%%%%%%%%%%%%%%
        \subsubsection{Power Results} %%%%%%%%%%%%%%%%%%%%%%%%%%%%%%%%%%%

        %%%%%%%%%% Power GRAPH %%%%%%%%%%
        \begin{figure}[hbt!]
            \centering
            \noindent\begin{subfigure}{1\textwidth}
            \includegraphics[width=1\textwidth]{images/Power/raytrace.PNG}
            % \caption{}
            \end{subfigure}% 

            \caption{Processor power for various PHT sizes.}
            \label{appfig:power:raytrace}
        \end{figure}
        % \clearpage
        %%%%%%%%%%%%%%%%%%%%%%%%%%%%%%%%%%%

        %%%%%%%%%% Power VALUES %%%%%%%%%%
        \begin{figure}[hbt!]
            \centering
            \noindent\begin{subfigure}{0.75\textwidth}
            \lstinputlisting{../output/raytrace/0/power.out}
            \caption{default}
            \end{subfigure}%

            \noindent\begin{subfigure}{0.75\textwidth}
            \lstinputlisting{../output/raytrace/4/power.out}
            \caption{$2^{4}$ PHT size}
            \end{subfigure}%
        \end{figure}
        \clearpage

        \begin{figure}[hbt!]\ContinuedFloat
            \centering
            \noindent\begin{subfigure}{0.75\textwidth}
            \lstinputlisting{../output/raytrace/8/power.out}
            \caption{$2^{8}$ PHT size}
            \end{subfigure}%

            \noindent\begin{subfigure}{0.75\textwidth}
            \lstinputlisting{../output/raytrace/16/power.out}
            \caption{$2^{16}$ PHT size}
            \end{subfigure}%

            \caption{Specific values for each components' power consumption.}
            \label{appfig:power:raytrace:values}
        \end{figure}
        \clearpage
        %%%%%%%%%%%%%%%%%%%%%%%%%%%%%%%%%%%%%%%%%%%%%%%%%%%%%%%%%%%%%%%%%%%%%%

        \subsubsection{CPI Stacks} %%%%%%%%%%%%%%%%%%%%%%%%%%%%%%%%%%%

        %%%%%%%%%% CPI GRAPH %%%%%%%%%%
        \begin{figure}[hbt!]
            \centering
            \noindent\begin{subfigure}{0.8\textwidth}
            \includegraphics[width=1\textwidth]{../output/raytrace/0/cpi-stack.png}
            \caption{default}
            \end{subfigure}%

            \noindent\begin{subfigure}{0.8\textwidth}
            \includegraphics[width=1\textwidth]{../output/raytrace/0/cpi-stack.png}
            \caption{$2^{4}$ PHT size}
            \end{subfigure}%
        \end{figure}
        \clearpage

        \begin{figure}[hbt!]\ContinuedFloat
            \centering
            \noindent\begin{subfigure}{0.8\textwidth}
            \includegraphics[width=1\textwidth]{../output/raytrace/8/cpi-stack.png}
            \caption{$2^{8}$ PHT size}
            \end{subfigure}%

            \noindent\begin{subfigure}{0.8\textwidth}
            \includegraphics[width=1\textwidth]{../output/raytrace/16/cpi-stack.png}
            \caption{$2^{16}$ PHT size}
            \end{subfigure}%

            \caption{CPI stacks for various PHT sizes.}
            \label{appfig:cpi:raytrace}
        \end{figure}
        \clearpage
        %%%%%%%%%%%%%%%%%%%%%%%%%%%%%%%%%%%

        %%%%%%%%%% CPI VALUES %%%%%%%%%%
        \begin{figure}[hbt!]
            \centering
            \noindent\begin{subfigure}{0.75\textwidth}
            \lstinputlisting{../output/raytrace/0/cpi-stack.out}
            \caption{default}
            \end{subfigure}%

            \noindent\begin{subfigure}{0.75\textwidth}
            \lstinputlisting{../output/raytrace/4/cpi-stack.out}
            \caption{$2^{4}$ PHT size}
            \end{subfigure}%
        \end{figure}
        \clearpage

        \begin{figure}[hbt!]\ContinuedFloat
            \centering
            \noindent\begin{subfigure}{0.75\textwidth}
            \lstinputlisting{../output/raytrace/8/cpi-stack.out}
            \caption{$2^{8}$ PHT size}
            \end{subfigure}%

            \noindent\begin{subfigure}{0.75\textwidth}
            \lstinputlisting{../output/raytrace/16/cpi-stack.out}
            \caption{$2^{16}$ PHT size}
            \end{subfigure}%

            \caption{Specific values for each components' CPI stack.}
            \label{appfig:cpi:raytrace:values}
        \end{figure}
        \clearpage
        %%%%%%%%%%%%%%%%%%%%%%%%%%%%%%%%%%%%%%%%%%%%%%%%%%%%%%%%%%%%%%%%%%%%%%

\end{document}