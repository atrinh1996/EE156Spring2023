%%%%%%%%%%%%%%%%%%%%%%%%%%%%%%%%%%%%%%%%%%%%%%%%%%%%%%%%%%%%%%%%%%%%%%%%
%
% Template latex file for a common article class for class notes
% and write ups. Additional Configuration and styling options are 
% commented out. ex. Table of Contents and Title page
% 
% Author: Amy Bui
% 
%%%%%%%%%%%%%%%%%%%%%%%%%%%%%%%%%%%%%%%%%%%%%%%%%%%%%%%%%%%%%%%%%%%%%%%%
 
\documentclass[12pt]{article}
\usepackage[utf8]{inputenc}
\usepackage{parskip}
\usepackage{tabularx}
\usepackage{appendix}
% \usepackage[color, leftbars]{changebar}

% Important Configurations
 
%%%%%%%%%%%%%%%%%%%%%%%%%%%%%%%%%%%%%%%%%%%%%%%%%%%%%%%%%%%%%%%%%%%%%%%%
% Reduce margin
%
% \addtolength{\oddsidemargin}{-.85in}
% \addtolength{\evensidemargin}{-.85in}
% \addtolength{\textwidth}{1in}

% \addtolength{\topmargin}{-.85in}
% \addtolength{\textheight}{1in}

% Page format commands:
% Override normal article margins,
% making the margins smaller
\setlength{\textwidth}{6.5in}
\setlength{\textheight}{9in}
\setlength{\oddsidemargin}{0in}
\setlength{\evensidemargin}{0in}
\setlength{\topmargin}{-0.5in}

\setlength{\parindent}{0pt}
%%%%%%%%%%%%%%%%%%%%%%%%%%%%%%%%%%%%%%%%%%%%%%%%%%%%%%%%%%%%%%%%%%%%%%%%


%%%%%%%%%%%%%%%%%%%%%%%%%%%%%%%%%%%%%%%%%%%%%%%%%%%%%%%%%%%%%%%%%%%%%%%%
% Math Symbols
\usepackage{mathtools}
\usepackage{amssymb}
% \usepackage{epsfig}
\usepackage{amsmath,amsthm}
\usepackage{amscd,amsxtra,latexsym}


% add floor and ceiling symbol. Usage: \ceil*{}, \floor*{}
\DeclarePairedDelimiter\ceil{\lceil}{\rceil}
\DeclarePairedDelimiter\floor{\lfloor}{\rfloor}

% multiset \langle ... \rangle
\def\multiset#1#2{\ensuremath{\left(\kern-.3em\left(\genfrac{}{}{0pt}{}{#1}{#2}\right)\kern-.3em\right)}}



%%%%%%%%%%%%%%%%%%%%%%%%%%%%%%%%%%%%%%%%%%%%%%%%%%%%%%%%%%%%%%%%%%%%%%%%

%%%%%%%%%%%%%%%%%%%%%%%%%%%%%%%%%%%%%%%%%%%%%%%%%%%%%%%%%%%%%%%%%%%%%%%%
% Code Sample Styling

% use \lstinline! xxx ! or \begin{lstlisting} ... \end{lstlisting}
\usepackage{listings}

\usepackage{color}
\definecolor{light-gray}{gray}{0.97} % shade of grey
\definecolor{dkgreen}{rgb}{0,0.6,0}
\definecolor{gray}{rgb}{0.5,0.5,0.5}
\definecolor{mauve}{rgb}{0.58,0,0.82}

% \begin{lstlisting}[...] ... \end{lstlisting}
\lstset{frame=none,
    language=Verilog,
    aboveskip=3mm,
    belowskip=3mm,
    stepnumber=1, % set to 0 if you don't like line nums
    showstringspaces=false,
    columns=flexible,
    basicstyle={\small\ttfamily},
    numbers=left,
    numberstyle=\color{black},
    keywordstyle=\color{blue},
    commentstyle=\color{dkgreen},
    stringstyle=\color{mauve},
    backgroundcolor=\color{light-gray},
    breaklines=true,
    breakatwhitespace=false,
    tabsize=2
}



%%%%%%%%%%%%%%%%%%%%%%%%%%%%%%%%%%%%%%%%%%%%%%%%%%%%%%%%%%%%%%%%%%%%%%%%

%%%%%%%%%%%%%%%%%%%%%%%%%%%%%%%%%%%%%%%%%%%%%%%%%%%%%%%%%%%%%%%%%%%%%%%%
\usepackage{xcolor}
%% https://tex.stackexchange.com/questions/401750/quick-and-short-command-for-coloring-one-word
\newcommand\shorthandon{\catcode`@=\active \catcode`^=\active \catcode`*=\active }
\newcommand\shorthandoff{\catcode`@=12 \catcode`^=7 \catcode`*=12 }
\shorthandon
\def@#1@{\textcolor{red}{#1}}%
\def^#1^{\textcolor{blue}{#1}}%
\def*#1{\string#1}
\shorthandoff
%% useage: \textcolor{red}{text here}
% \shorthandon
% This is a @test@ of the ^emergency^ bro*@dcast system.
% \shorthandoff
%%%%%%%%%%%%%%%%%%%%%%%%%%%%%%%%%%%%%%%%%%%%%%%%%%%%%%%%%%%%%%%%%%%%%%%%


%%%%%%%%%%%%%%%%%%%%%%%%%%%%%%%%%%%%%%%%%%%%%%%%%%%%%%%%%%%%%%%%%%%%%%%%

%Commands below change page margins (this much space at the titlepage, etc)
\newlength{\toppush}
\setlength{\toppush}{2\headheight}
\addtolength{\toppush}{\headsep}

% Section header Styling
% The commands below change the bold text where it says "Section" into "Question"
% \usepackage{titlesec}
% \titleformat{\section}
% {\normalfont\Large\bfseries}{Question~\thesection:}{1em}{}

% I added this command below to chance "subsections numbers" to be "Question [subsection number]" -AB 1/31/2021
% \titleformat{\subsection}
% {\normalfont\bfseries}{\thesubsection:}{1em}{}

% Page head Styling
% Name and subject of the class
\def\subjnum{EE 156}          % Class Number
\def\subjname{Advance Topics in Computer Architecture}       % Class Name

% Name of the student, university name and which semester
\def\doheading#1#2#3{\vfill\eject\vspace*{-\toppush}%
  \vbox{\hbox to\textwidth{{\bf} \subjnum: \subjname \hfil Amy Bui}%
    \hbox to\textwidth{{\bf} Tufts University, Spring 2023 \hfil#3\strut}%
    \hrule}}

%Command for the title of the document (Homework 0)
\newcommand{\htitle}[1]{\vspace*{1.25ex plus 1ex minus 0ex}%
\begin{center}
    {\large\bf #1}
\end{center}} 
%%%%%%%%%%%%%%%%%%%%%%%%%%%%%%%%%%%%%%%%%%%%%%%%%%%%%%%%%%%%%%%%%%%%%%%%



%%%%%%%%%%%%%%%%%%%%%%%%%%%%%%%%%%%%%%%%%%%%%%%%%%%%%%%%%%%%%%%%%%%%%%%%
% Misc
\usepackage{graphicx} % graphics
\usepackage{enumitem} % listing style (bullet lists)

% below helps with trying to get figures in a row
\usepackage{caption}
\usepackage{subcaption}

% hyperlink styling
% use \href{} and \url{}, and colors table of contents links
% use \href{} and \url{}
% \label{sec:name}
% \hyperref[label]{text}
\usepackage{hyperref}
\hypersetup{
    colorlinks=true,
    linkcolor=blue, % was previously black
    filecolor=magenta,
    urlcolor=blue,
    pdftitle={Template}
}
\urlstyle{same}

% A command for primes (')
\newcommand{\p}%
    {\ensuremath{^{\prime}}}

% a command for double primes ('')
\newcommand{\pp}%
    {\ensuremath{^{\prime \prime}}}

% A command for the Kleene star
\newcommand{\str}%
    {\ensuremath{^{\star}}}

% a command for the double star
\newcommand{\sstr}%
    {\ensuremath{^{\star\star}}}
%%%%%%%%%%%%%%%%%%%%%%%%%%%%%%%%%%%%%%%%%%%%%%%%%%%%%%%%%%%%%%%%%%%%%%%%

% Options for title page, use \maketitle in document
% \author{Amy Bui}
% \title{COMP160 - Algorithms: Class Notes and Practice}

\begin{document}
%% create title page
% \title{(g)ROOT \\ Language Reference Manual}
% \author{Samuel Russo \quad Amy Bui \quad Eliza Encherman \\ Zachary Goldstein \quad Nickolas Gravel}
% \date{\today}
% \maketitle

\doheading{2}{title}{Lab 0}

    %%%%%%%%%%%%%%%%%%%%%%%%%%%%%%%%%%%%%%%%%%%%%%%%%%%%%%%%%%%%%%%%%%%%%%%%
    % Table of Contents
    \setcounter{tocdepth}{2}
    \tableofcontents
    % \pagebreak
    %%%%%%%%%%%%%%%%%%%%%%%%%%%%%%%%%%%%%%%%%%%%%%%%%%%%%%%%%%%%%%%%%%%%%%%%

    \section{Experimental Setup}

        \subsection{Benchmarks}
        \begin{itemize}
            \item fft
            \item fft 
            \item fft
        \end{itemize}

        \subsection{\textsc{Input Size}}: 

        \subsection{\textsc{Configuration Files}}
        \begin{itemize}
            \item ffile
        \end{itemize}

        \subsection{\textsc{Architecture Configuration}} (figure)

    %%%%%%%%%%%%%%%%%%%%%%%%%%%%%%%%%%%%%%%%%%%%%%%%%%%%%%%%%%%%%%%%%%%%%%%%


    \section{Energy Results}


    %%%%%%%%%%%%%%%%%%%%%%%%%%%%%%%%%%%%%%%%%%%%%%%%%%%%%%%%%%%%%%%%%%%%%%%%


    \section{Performance Analysis}




    % \begin{figure}[h]
    %     \centering
    %     \includegraphics[width=0.7\linewidth]{images/and2_wave.png}
    %     \caption{and2 waveform given by my testbench in App. \ref{app:and2tb}. Input \textsf{in0} and \textsf{in1} initially set to \textsf{unknown}, then every 100ns, they are set to the values specified in \lstinline!and2_tb.vhd!. See Fig \ref{fig:and2waveresults}.}
    %     \label{fig:and2wave}
    % \end{figure}

    % \begin{figure}[h]
    %     \centering
    %     \begin{tabular}{cc}
    %         \begin{subfigure}{.5\textwidth}
    %             \centering
    %             \includegraphics[width=1\textwidth]{images/and2_wave000.png}
    %             \caption{\textsf{0 \& 0 = 0}}
    %             \label{fig:and2wave000}
    %         \end{subfigure} &
    %         \begin{subfigure}{.5\textwidth}
    %             \centering
    %             \includegraphics[width=1\textwidth]{images/and2_wave010.png}
    %             \caption{\textsf{0 \& 1 = 0}}
    %             \label{fig:and2wave010}
    %         \end{subfigure} \\
    %         \begin{subfigure}{.5\textwidth}
    %             \centering
    %             \includegraphics[width=1\textwidth]{images/and2_wave100.png}
    %             \caption{\textsf{1 \& 0 = 0}}
    %             \label{fig:and2wave100}
    %         \end{subfigure} &
    %         \begin{subfigure}{.5\textwidth}
    %             \centering
    %             \includegraphics[width=1\textwidth]{images/and2_wave111.png}
    %             \caption{\textsf{1 \& 1 = 1}}
    %             \label{fig:and2wave111}
    %         \end{subfigure}
    %     \end{tabular}
    %     \caption{Individual testbench results (test cases marked by vertical line through each 100ns cycle and truth table row in App. \ref{app:truth}). Each figure is labeled with the values formated as \lstinline!<in0> \& <in1> = <output>!}
    %     \label{fig:and2waveresults}
    % \end{figure}

  


    % \begin{thebibliography}{1}
    %     \bibitem[1]{ghdl}\href{http://ghdl.free.fr/site/pmwiki.php?n=Main.Installation}{GHDL and GTKWave binaries (free)}
    %     \bibitem[2]{ghdlvid}\href{https://www.youtube.com/watch?v=dvLeDNbXfFw}{GHDL and GTKWave: Getting Started} 
    %     \bibitem[3]{oh}Mark and Parnian's office hours
    % \end{thebibliography}
    % \pagebreak

\end{document}