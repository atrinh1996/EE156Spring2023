%%%%%%%%%%%%%%%%%%%%%%%%%%%%%%%%%%%%%%%%%%%%%%%%%%%%%%%%%%%%%%%%%%%%%%%%
%
% Template latex file for a common article class for class notes
% and write ups. Additional Configuration and styling options are 
% commented out. ex. Table of Contents and Title page
% 
% Author: Amy Bui
% 
%%%%%%%%%%%%%%%%%%%%%%%%%%%%%%%%%%%%%%%%%%%%%%%%%%%%%%%%%%%%%%%%%%%%%%%%
 
\documentclass[12pt]{article}
\usepackage[utf8]{inputenc}
\usepackage{parskip}
\usepackage{tabularx}
\usepackage{array}
\usepackage{appendix}
% \usepackage[showframe=true]{geometry}
\usepackage{changepage}
% \usepackage{csvsimple}
% \usepackage[framemethod=tikz]{mdframed}
% \usepackage[color, leftbars]{changebar}
% \usepackage[inkscapeformat=png]{svg}
% \usepackage{svg}
% \usepackage[inkscape={/Applications/Inkscape.app/Contents/Resources/bin/inkscape -z -C}]{svg}

% Important Configurations
 
%%%%%%%%%%%%%%%%%%%%%%%%%%%%%%%%%%%%%%%%%%%%%%%%%%%%%%%%%%%%%%%%%%%%%%%%
% Reduce margin
%
% \addtolength{\oddsidemargin}{-.85in}
% \addtolength{\evensidemargin}{-.85in}
% \addtolength{\textwidth}{1in}

% \addtolength{\topmargin}{-.85in}
% \addtolength{\textheight}{1in}

% Page format commands:
% Override normal article margins,
% making the margins smaller
\setlength{\textwidth}{6.5in}
\setlength{\textheight}{9in}
\setlength{\oddsidemargin}{0in}
\setlength{\evensidemargin}{0in}
\setlength{\topmargin}{-0.6in}

\setlength{\parindent}{0pt}
%%%%%%%%%%%%%%%%%%%%%%%%%%%%%%%%%%%%%%%%%%%%%%%%%%%%%%%%%%%%%%%%%%%%%%%%


%%%%%%%%%%%%%%%%%%%%%%%%%%%%%%%%%%%%%%%%%%%%%%%%%%%%%%%%%%%%%%%%%%%%%%%%
% Math Symbols
\usepackage{mathtools}
\usepackage{amssymb}
% \usepackage{epsfig}
\usepackage{amsmath,amsthm}
\usepackage{amscd,amsxtra,latexsym}


% add floor and ceiling symbol. Usage: \ceil*{}, \floor*{}
\DeclarePairedDelimiter\ceil{\lceil}{\rceil}
\DeclarePairedDelimiter\floor{\lfloor}{\rfloor}

% multiset \langle ... \rangle
\def\multiset#1#2{\ensuremath{\left(\kern-.3em\left(\genfrac{}{}{0pt}{}{#1}{#2}\right)\kern-.3em\right)}}



%%%%%%%%%%%%%%%%%%%%%%%%%%%%%%%%%%%%%%%%%%%%%%%%%%%%%%%%%%%%%%%%%%%%%%%%

%%%%%%%%%%%%%%%%%%%%%%%%%%%%%%%%%%%%%%%%%%%%%%%%%%%%%%%%%%%%%%%%%%%%%%%%
% Code Sample Styling

% use \lstinline! xxx ! or \begin{lstlisting} ... \end{lstlisting}
\usepackage{listings}

\usepackage{color}
\definecolor{light-gray}{gray}{0.97} % shade of grey
\definecolor{dkgreen}{rgb}{0,0.6,0}
\definecolor{gray}{rgb}{0.5,0.5,0.5}
\definecolor{mauve}{rgb}{0.58,0,0.82}

% \begin{lstlisting}[...] ... \end{lstlisting}
\lstset{frame=none,
    language=Verilog,
    aboveskip=3mm,
    belowskip=3mm,
    stepnumber=0, % set to 0 if you don't like line nums
    showstringspaces=false,
    columns=flexible,
    basicstyle={\small\ttfamily},
    numbers=left,
    numberstyle=\color{black},
    keywordstyle=\color{blue},
    commentstyle=\color{dkgreen},
    stringstyle=\color{mauve},
    backgroundcolor=\color{light-gray},
    breaklines=true,
    breakatwhitespace=false,
    tabsize=2
}

% \newcommand\mylstcaption{}

% \mdfdefinestyle{mymdstyle}{
% hidealllines=true,
% middleextra={
%   \node[anchor=west] at (O|-P)
%     {\lstlistingname~\thelstlisting\  (Cont.):~\mylstcaption};},
% secondextra={
%   \node[anchor=west] at (O|-P)
%     {\lstlistingname~\thelstlisting\  (Cont.):~\mylstcaption};},
% splittopskip=2\baselineskip
% }

% \surroundwithmdframed[style=mymdstyle]{lstlisting}
% \newmdenv[style=mymdstyle]{mdlisting}



%%%%%%%%%%%%%%%%%%%%%%%%%%%%%%%%%%%%%%%%%%%%%%%%%%%%%%%%%%%%%%%%%%%%%%%%

%%%%%%%%%%%%%%%%%%%%%%%%%%%%%%%%%%%%%%%%%%%%%%%%%%%%%%%%%%%%%%%%%%%%%%%%
\usepackage{xcolor}
%% https://tex.stackexchange.com/questions/401750/quick-and-short-command-for-coloring-one-word
\newcommand\shorthandon{\catcode`@=\active \catcode`^=\active \catcode`*=\active }
\newcommand\shorthandoff{\catcode`@=12 \catcode`^=7 \catcode`*=12 }
\shorthandon
\def@#1@{\textcolor{red}{#1}}%
\def^#1^{\textcolor{blue}{#1}}%
\def*#1{\string#1}
\shorthandoff
%% useage: \textcolor{red}{text here}
% \shorthandon
% This is a @test@ of the ^emergency^ bro*@dcast system.
% \shorthandoff
%%%%%%%%%%%%%%%%%%%%%%%%%%%%%%%%%%%%%%%%%%%%%%%%%%%%%%%%%%%%%%%%%%%%%%%%


%%%%%%%%%%%%%%%%%%%%%%%%%%%%%%%%%%%%%%%%%%%%%%%%%%%%%%%%%%%%%%%%%%%%%%%%

%Commands below change page margins (this much space at the titlepage, etc)
\newlength{\toppush}
\setlength{\toppush}{2\headheight}
\addtolength{\toppush}{\headsep}

% Section header Styling
% The commands below change the bold text where it says "Section" into "Question"
% \usepackage{titlesec}
% \titleformat{\section}
% {\normalfont\Large\bfseries}{Question~\thesection:}{1em}{}

% I added this command below to chance "subsections numbers" to be "Question [subsection number]" -AB 1/31/2021
% \titleformat{\subsection}
% {\normalfont\bfseries}{\thesubsection:}{1em}{}

% Page head Styling
% Name and subject of the class
\def\subjnum{EE 156}          % Class Number
\def\subjname{Advance Topics in Computer Architecture}       % Class Name

% Name of the student, university name and which semester
\def\doheading#1#2#3{\vfill\eject\vspace*{-\toppush}%
  \vbox{\hbox to\textwidth{{\bf} \subjnum: \subjname \hfil Amy Bui}%
    \hbox to\textwidth{{\bf} Tufts University, Spring 2023 \hfil#3\strut}%
    \hrule}}

%Command for the title of the document (Homework 0)
\newcommand{\htitle}[1]{\vspace*{1.25ex plus 1ex minus 0ex}%
\begin{center}
    {\large\bf #1}
\end{center}} 
%%%%%%%%%%%%%%%%%%%%%%%%%%%%%%%%%%%%%%%%%%%%%%%%%%%%%%%%%%%%%%%%%%%%%%%%



%%%%%%%%%%%%%%%%%%%%%%%%%%%%%%%%%%%%%%%%%%%%%%%%%%%%%%%%%%%%%%%%%%%%%%%%
% Misc
\usepackage{graphicx} % graphics
\usepackage{enumitem} % listing style (bullet lists)

% below helps with trying to get figures in a row
\usepackage{caption}
\usepackage{subcaption}

% hyperlink styling
% use \href{} and \url{}, and colors table of contents links
% use \href{} and \url{}
% \label{sec:name}
% \hyperref[label]{text}
\usepackage{hyperref}
\hypersetup{
    colorlinks=true,
    linkcolor=blue, % was previously black
    filecolor=magenta,
    urlcolor=blue,
    pdftitle={Template}
}
\urlstyle{same}

% A command for primes (')
\newcommand{\p}%
    {\ensuremath{^{\prime}}}

% a command for double primes ('')
\newcommand{\pp}%
    {\ensuremath{^{\prime \prime}}}

% A command for the Kleene star
\newcommand{\str}%
    {\ensuremath{^{\star}}}

% a command for the double star
\newcommand{\sstr}%
    {\ensuremath{^{\star\star}}}
%%%%%%%%%%%%%%%%%%%%%%%%%%%%%%%%%%%%%%%%%%%%%%%%%%%%%%%%%%%%%%%%%%%%%%%%

% Options for title page, use \maketitle in document
% \author{Amy Bui}
% \title{COMP160 - Algorithms: Class Notes and Practice}


\begin{document}
%% create title page
% \title{(g)ROOT \\ Language Reference Manual}
% \author{Samuel Russo \quad Amy Bui \quad Eliza Encherman \\ Zachary Goldstein \quad Nickolas Gravel}
% \date{\today}
% \maketitle

\doheading{2}{title}{Lab 1}

    %%%%%%%%%%%%%%%%%%%%%%%%%%%%%%%%%%%%%%%%%%%%%%%%%%%%%%%%%%%%%%%%%%%%%%%%
    % Table of Contents
    % \setcounter{tocdepth}{2}
    % \tableofcontents
    % \pagebreak
    %%%%%%%%%%%%%%%%%%%%%%%%%%%%%%%%%%%%%%%%%%%%%%%%%%%%%%%%%%%%%%%%%%%%%%%%

    % \begin{thebibliography}{1}
    %     \bibitem[1]{sniper}\href{https://snipersim.org/w/The_Sniper_Multi-Core_Simulator}{The Sniper Multi-Core Simulator}
    %     \bibitem[2]{parallel}O. Tange (2011): \href{https://www.gnu.org/software/parallel/parallel_tutorial.html}{GNU Parallel}  - The Command-Line Power Tool
    %     \bibitem[3]{splash2}S. C. Woo, M. Ohara, E. Torrie, J. P. Singh and A. Gupta, \href{https://citeseerx.ist.psu.edu/viewdoc/download?doi=10.1.1.48.2356&rep=rep1&type=pdf}{The SPLASH-2 Programs: Characterizaion and Methodological Considerations}, Proceedings 22nd Annual International Symposium on Computer Architecture, Santa Margherita Ligure, Italy, 1995, pp. 24-36
    %     \bibitem[4]{npb}Bailey DH, Barszcz E, Barton JT, et al. \href{https://www.nas.nasa.gov/software/npb.html}{The Nas Parallel Benchmarks}. The International Journal of Supercomputing Applications. 1991;5(3):63-73. doi:\url{10.1177/109434209100500306}
    %     \bibitem[5]{book}John L. Hennessy and David A. Patterson. 2017. Computer Architecture, Sixth Edition: A Quantitative Approach (6th. ed.). Morgan Kaufmann Publishers Inc., San Francisco, CA, USA.
    % \end{thebibliography}
    % \clearpage

    \bibliographystyle{acm}%Used BibTeX style is acm
    \bibliography{sources}

    \section{Hypothesis}
    \label{intro}

        % Chapter 3.4 (Dynamic Scheduling, Tomasulo's Algorithm) \cite{textbook,sniper,splash2,npb,parallel,ua} 

        Increasing the number of reservation stations and the commit width will increase IPC. By increasing the instruction-level parallelism (ILP),  this reduces data hazards and control hazard stalls, and therefore improves performance \cite{textbook}.

    % \clearpage
    %%%%%%%%%%%%%%%%%%%%%%%%%%%%%%%%%%%%%%%%%%%%%%%%%%%%%%%%%%%%%%%%%%%%%%%%
    

    \section{Experiment Design}
    \label{sec:setup}

        \begin{table}[h]
        \begin{center} 
            \begin{tabular}{c||c||c}
                \begin{tabular}{|l|}
                    \hline
                    \textbf{Benchmark} \\ 
                    \hline 
                    \hline
                    \texttt{splash2-ocean.cont} \\ 
                    \texttt{splash2-radix}\\
                    \texttt{splash2-barnes}\\
                    \texttt{npb-is}\\
                    \texttt{npb-ep}\\
                    \texttt{npb-mg}\\
                    \texttt{npb-ua}\\
                    \hline 
                \end{tabular}
                & 
                \begin{tabular}{|l|}
                    \hline
                    \textbf{ROB: commit\_width} \\ 
                    \hline 
                    \hline
                    36 \\
                    \hline 
                \end{tabular}
                &
                \begin{tabular}{|l|}
                    \hline
                    \textbf{Reservation Station Entries} \\ 
                    \hline 
                    \hline
                    32 \\ 
                    64 \\
                    128 \\
                    512 \\
                    1024 \\
                    \hline 
                \end{tabular}
            \end{tabular}
            \caption{Configuration parameters and values swept in the experiment.}
            \label{table:configurations}
        \end{center}
        \end{table}


\end{document}