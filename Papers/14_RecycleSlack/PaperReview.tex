\documentclass [12pt]{article}
\usepackage{epsfig}
\usepackage{enumitem}
\usepackage{amsmath}
% \usepackage[color, leftbars]{changebar}
% \usepackage{fontawesome} 
% \usepackage{caption}
% \usepackage{subcaption}


\setlength{\textwidth}{6.5in}
\setlength{\textheight}{9in}
\setlength{\oddsidemargin}{0in}
\setlength{\evensidemargin}{0in}
\setlength{\topmargin}{-0.5in}

\setlength{\parindent}{0pt}

% \newtheorem{theorem}{Theorem}[section]
% \newtheorem{definition}[theorem]{Definition}
% \newtheorem{claim}[theorem]{Claim}
% \newtheorem{lemma}[theorem]{Lemma}
% \newtheorem{proof}[theorem]{Proof}

\newlength{\toppush}
\setlength{\toppush}{2\headheight}
\addtolength{\toppush}{\headsep}

\usepackage{hyperref}
\hypersetup{
    colorlinks=true,
    linkcolor=blue, % was previously black
    filecolor=magenta,
    urlcolor=blue,
    pdftitle={Template}
}
\urlstyle{same}


\def\subjnum{EE 156}
\def\subjname{Adv. Comp. Arch.}

\def\doheading#1#2#3{\vfill\eject\vspace*{-\toppush}%
  \vbox{\hbox to\textwidth{{\bf} \subjnum: \subjname \hfil Amy Bui}%
    \hbox to\textwidth{{\bf} Tufts University, Spring 2023 \hfil#3\strut}%
    \hrule}}

\newcommand{\htitle}[1]{\vspace*{3.25ex plus 1ex minus .2ex}%
\begin{center}
{\large\bf #1}
\end{center}} 

%%%%%%%%%%%%%%%%%%%%%%%%%%%%%%%%%%%%%%%%%%%%%%%%%%%%%%%%%%%%%%%%%%%

\begin{document}
\doheading{2}{title}{Paper Review} 
% \htitle{Paper Info}
% \bigskip 
% \bigskip 
%%%%%%%%%% begin text after this line %%%%%%%%%%%%%%

    %%%%%%%%%%%%%%%%%%%%%%%%%%%%%%%%%%%%%%%%%%%%%%%%%%%%%%%%%%%%%%%%%%%%%%%%%
    \section{Summary}
    \label{sec:summary}

        \textbf{Recycling Data Slack in Out-of-Order Cores (2019):} 
        
        The authors propose a mechanism called ReDSOC that dynamically identifies data slack and recycles it efficiently, improving performance even in OOO cores. Microarchitecture designes can sacrifice energy efficiency and performance in favor of reliablility through more complex ISA semantics and the conservative time margins for clock cycles that accommodate them, which itself leads to clock cycle slack when nonusefule work is done; data slack, in particular, is from the inactive critical paths, is data dependent, and varies widely, potentially wasting half a clock cycle. The authors detail how the different aspects of ReDSOC are implemented in general OOO cores and show that ReDSOC achieves application speedups (about 5\%-25\%) and can even outperform competing mechanisms.

    
        
  
        

    %%%%%%%%%%%%%%%%%%%%%%%%%%%%%%%%%%%%%%%%%%%%%%%%%%%%%%%%%

    \section{Strengths} %%%%%%%%%%%%%%%%%%%%%%
    \label{sec:strengths}
        \begin{itemize}
            \item The section 2 intro and Figure 1 both support the authors' data slack classifications with qualifying quantitative observations, which helps justify the descriptive categories of data slack according to slack sources. The three categories of data slack themselves were also explained thoroughly.  
            \item The authors tested their ReDSOC implementation thoroughly across some variety of processors and applications, rather than with a narrow breadth of testing that was seen in the last few papers, and even included some comparisons of ReDSOC against competing proposals to address data slack. 
        \end{itemize}
    %%%%%%%%%%%%%%%%%%%%%%%%%%%%%%%%%%%%%%%%%%%%%%%%%%%%%%%%%

    \section{Weaknesses} %%%%%%%%%%%%%%%%%%%%%%
    \label{sec:weaknesses}
        \begin{itemize}
            \item Section 2's topic on slack estimation would improve with the inclusion of a figure to visualize how their data-width predictor works or fits into the overall mechanism.
            \item Section 4 has a few instances of in-depth example walk-throughs of the slack-aware scheduling mechanism that make this discussion section more long-winded than necessary. The higher-level explanations, key takeaways, and even specific problems they are trying to address can get lost. Section 4 could be split into 2+ sections or the walk-throughs could be done in an appendix. 
        \end{itemize}
    %%%%%%%%%%%%%%%%%%%%%%%%%%%%%%%%%%%%%%%%%%%%%%%%%%%%%%%%%

    \section{Rating: 4} %%%%%%%%%%%%%%%%%%%%%% 
    \label{sec:rating}
    % \pagebreak
    %%%%%%%%%%%%%%%%%%%%%%%%%%%%%%%%%%%%%%%%%%%%%%%%%%%%%%%%%

    \section{Comments} %%%%%%%%%%%%%%%%%%%%%%
    \label{sec:comments}

    This paper is 


    %%%%%%%%%%%%%%%%%%%%%%%%%%%%%%%%%%%%%%%%%%%%%%%%%%%%%%%%%
 
        
    % \pagebreak
    % END %%%%%%%%%%%%%%%%%%%%%%%%%%%%%%%%%%%%%%%%%%%%%%%%%%%%%%%%%%%%%%%%%%%

    % \section{Notes}
    %     \begin{itemize}
    %         \item 
    %     \end{itemize}

% \begin{thebibliography}{1}
%     \bibitem[1]{paper}G. S. Ravi and M. Lipasti, "Recycling Data Slack in Out-of-Order Cores," 2019 IEEE International Symposium on High Performance Computer Architecture (HPCA), Washington, DC, USA, 2019, pp. 545-557, doi: 10.1109/HPCA.2019.00065. https://ieeexplore.ieee.org/document/8675213
% \end{thebibliography}
%%%%%%%%%%%%%%%%%%%%%%%%%%%%%%%%%%%%%%%%%%%%%%%%%%%%%%%%%%%%%%%%%%%%%%
\end{document}
%%%%%%%%%%%%%%%%%%%%%%%%%%%%%%%%%%%%%%%%%%%%%%%%%%%%%%%%%%%%%%%%%%%%%%

