\documentclass [12pt]{article}
\usepackage{epsfig}
\usepackage{enumitem}
\usepackage{amsmath}
% \usepackage[color, leftbars]{changebar}
% \usepackage{fontawesome} 
% \usepackage{caption}
% \usepackage{subcaption}


\setlength{\textwidth}{6.5in}
\setlength{\textheight}{9in}
\setlength{\oddsidemargin}{0in}
\setlength{\evensidemargin}{0in}
\setlength{\topmargin}{-0.5in}

\setlength{\parindent}{0pt}

% \newtheorem{theorem}{Theorem}[section]
% \newtheorem{definition}[theorem]{Definition}
% \newtheorem{claim}[theorem]{Claim}
% \newtheorem{lemma}[theorem]{Lemma}
% \newtheorem{proof}[theorem]{Proof}

\newlength{\toppush}
\setlength{\toppush}{2\headheight}
\addtolength{\toppush}{\headsep}

\usepackage{hyperref}
\hypersetup{
    colorlinks=true,
    linkcolor=blue, % was previously black
    filecolor=magenta,
    urlcolor=blue,
    pdftitle={Template}
}
\urlstyle{same}


\def\subjnum{EE 156}
\def\subjname{Adv. Comp. Arch.}

\def\doheading#1#2#3{\vfill\eject\vspace*{-\toppush}%
  \vbox{\hbox to\textwidth{{\bf} \subjnum: \subjname \hfil Amy Bui}%
    \hbox to\textwidth{{\bf} Tufts University, Spring 2023 \hfil#3\strut}%
    \hrule}}

\newcommand{\htitle}[1]{\vspace*{3.25ex plus 1ex minus .2ex}%
\begin{center}
{\large\bf #1}
\end{center}} 

%%%%%%%%%%%%%%%%%%%%%%%%%%%%%%%%%%%%%%%%%%%%%%%%%%%%%%%%%%%%%%%%%%%

\begin{document}
\doheading{2}{title}{Generate Project Ideas} 
% \htitle{Paper Info}
% \bigskip 
% \bigskip 
%%%%%%%%%% begin text after this line %%%%%%%%%%%%%%

    %%%%%%%%%%%%%%%%%%%%%%%%%%%%%%%%%%%%%%%%%%%%%%%%%%%%%%%%%%%%%%%%%%%%%%%%%
    \begin{enumerate}
        \item Coherency correlates with qubit stability and extending qubit decoherence time is important in reducing memory errors in QC. For the project, we can explore sources of qubit decoherence, methods of mitigation, and survey recent research on mitigation. As an experiement, we could measure decoherence time as a function of number of qubits in a NISQ computer. 
        \item For the project, we can characterize different noise in a quantum system. Noise has an impact on both results and performance. As an experiment, we can look at how different noise models affect performance of available quantum algorithms and/or benchmarks, and explore techniques that mitigate versus ``correct'' for noise. 
        \item Quantum cryptography is still theoretical because the hardware is not yet available. Simulators are used to study and develop these encryption algorithms \cite{wang}. For a project, it would be interesting to survey the literature and simulate available protocols or design an experiement that measures how quantum cryptographic techniques hold up. 
    \end{enumerate}

    \emph{For running experiment, Qiskit appears to be a versatile option \cite{qiskit}.}

    %%%%%%%%%%%%%%%%%%%%%%%%%%%%%%%%%%%%%%%%%%%%%%%%%%%%%%%%%



\begin{thebibliography}{1}
    % \bibitem[1]{simulators}\href{https://quantiki.org/wiki/list-qc-simulators}{Quantiki}: List of QC Simulator and Descriptions.
    \bibitem[1]{qiskit}\href{https://qiskit.org/}{Qiskit: Quantum Simulator}
    \bibitem[2]{ding}Y. Ding, F. Chong. \href{https://link.springer.com/book/10.1007/978-3-031-01765-0}{Quantum Computer Systems: Research for Noisy Intermediate-Scale Quantum Computers}. Springer Cham. Synthesis Lectures on Computer Architecture. 2020
    \bibitem[3]{wang}Shuangbao Wang, Matthew Rohde, and Amjad Ali. 2020. Quantum Cryptography and Simulation: Tools and Techniques. In Proceedings of the 2020 4th International Conference on Cryptography, Security and Privacy (ICCSP 2020). Association for Computing Machinery, New York, NY, USA, 36-41. https://doi-org.ezproxy.library.tufts.edu/10.1145/3377644.3377671
\end{thebibliography}
%%%%%%%%%%%%%%%%%%%%%%%%%%%%%%%%%%%%%%%%%%%%%%%%%%%%%%%%%%%%%%%%%%%%%%
\end{document}
%%%%%%%%%%%%%%%%%%%%%%%%%%%%%%%%%%%%%%%%%%%%%%%%%%%%%%%%%%%%%%%%%%%%%%

